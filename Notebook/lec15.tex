\section{Anomaly Detection}
    \subsection{Motivation: Density Estimation}
        Generate a function \emph{p(x)}, such that $p(x) < \epsilon$ suggest anomaly. 
        The crux is to flag unusual behaviour, for example: 
        \begin{itemize}
            \item Fraud detection
            \item Manufacturing 
            \item Computer load in data center
        \end{itemize}

        \subsubsection{Guassian distribution}
            \begin{equation}
                x \sim \mathcal{N}(\mu, \sigma^2)
                \label{eq:guassian}
            \end{equation}

            \begin{itemize}
                \item $\mu$: mean
                \item $\sigma^2$: variance
            \end{itemize}

    \subsection{Algorithm}
        \begin{enumerate}
            \item Choose features $x_i$ that are indicative of anomalous examples. 
            \item Fit parameters 
                \begin{equation}
                \boldsymbol{\mu} = \begin{bmatrix}
                                            \mu_1       \\
                                            \vdots      \\
                                            \mu_i       \\
                                            \vdots      \\
                                            \mu_n       
                                        \end{bmatrix} = \frac{1}{m} \sum_{i=1}^{m} x_j^{(i)}
                    \label{eq:anomaly-mu}
                \end{equation} 
                                        and 
                \begin{equation}
                \boldsymbol{\sigma^2} = \begin{bmatrix}
                    \sigma^2_1       \\
                    \vdots      \\
                    \sigma^2_i       \\
                    \vdots      \\
                    \sigma^2_n       
                \end{bmatrix} = \frac{1}{m} \sum_{i=1}^{m} (x^{(i)} - \mu)^2
                    \label{eq:anomaly-sigma-squared}
                \end{equation}
        \item Given new example $x$, compute $p(x)$: 
                \begin{equation}
                    p(x) = \prod_{j=1}^{n} p(x_j;\mu_j, \sigma^2_j) = \prod_{j=1}^{n} \frac{1}{\sqrt{2 \pi} \sigma_j} exp(\frac{-(x_j - \mu_j)^2}{2 \mu_j ^2})
                    \label{eq:anomaly-p}
                \end{equation}

            Anomaly if $p(x) < \epsilon$

        \end{enumerate}
    \subsection{Developing and Evaluating an Anomaly Detection System}
        Dividing labelled data:
        \begin{itemize}
            \item Training set: 60\%.
            \item Cross validation set: 20\%
            \item Test set: 20\%.
        \end{itemize}
    
    \subsection{Anomaly Detection vs Supervised Learning}
    Refer to Table \ref{tab:anomaly_detection_vs_supervised_learning}. 

    \begin{table}[htpb]
            \centering
            \begin{tabular}{|p{0.5\linewidth}|p{0.5\linewidth}|}
                \hline
                Anomaly detection & Supervised learning \\
                \hline
                Small number of positive examples [y=1] (0-20) and large number of negative examples [y=0] & Large number of positive and negative examples \\
                \hline 
                Many different types of anomalies. Hard to learn anomalies from positive examples.  & Enough positive examples to learn positivity. \\
                \hline 
                Future anomalies may look nothing like previous anomalies & Future positive examples similar to training set.\\
                \hline
            \end{tabular}
            \caption{Anomaly detection vs Supervised learning}
            \label{tab:anomaly_detection_vs_supervised_learning}
        \end{table}
    
    \section{Choosing what features to use}
        \begin{itemize}
            \item \textbf{Goal}: Want $p(x)$ large for normal examples x; $p(x)$ small for anomalous examples x. 
            \item Ideally require Guassian features. 
            \item If not Guassian, require preprocessing, e.g. log(), $\sqrt{x}$, etc. 
            \item Most common problem: $p(x)$ is comparable for both normal and anomalous examples $\rightarrow$ come up with new features (e.g. ratios of current fetatures. )
        \end{itemize}
            
    \subsection{Multivariate Guassian Distribution}
        
    $x \in \mathbb{R}^n$. Model $p(x)$ with all dimensions in tandem. Parameters: $\boldsymbol{\mu} \in \mathbb{R}^n$, $\Sigma \in \mathbb{R}^{n \times n}$ (covariance matrix).
        
        \subsubsection{Anomaly detection using multivariate Guassian distribution}
        \begin{equation}
            p(x; \mu, \Sigma) = \frac{1}{(2\pi)^{n/2} |\Sigma|^{1/2}} exp (- \frac{1}{2} (x -\mu)^T \Sigma ^{-1} (x-\mu))        
            \label{eq:multivariate_anomaly_p}
        \end{equation}
        where 
            \begin{equation}
                \mu = \frac{1}{m} \sum_{i=1}^{m} x^{(i)}
                \label{eq:anomaly-mu-2}
            \end{equation}

            \begin{equation}
                \Sigma = \frac{1}{m} \sum_{i=1}^{m} (x^{(i)}-\mu) (x^{(i)}-\mu)^T
                \label{eq:}
            \end{equation}


        \subsubsection{Original model vs Multivariate Guassian}
            \begin{table}[htpb]
                \centering
                \begin{tabular}{|p{0.5\linewidth}|p{0.5\linewidth}|}
                    \hline
                    Original model & Multivariate Guassian \\
                    \hline
                    $\prod_{j=1}^{n} p(x_j;\mu_j, \sigma^2_j)$ (Eq. \ref{eq:anomaly-p} )& Eq. \ref{eq:multivariate_anomaly_p} \\
                    \hline
                    Manually combine features to capture anomalies & Automatically captures correlations between features. \\
                    \hline
                    Computationally cheaper and scales well & Computationally expensive (matrix) \\
                    \hline 
                    Works on small training set size (m) & Must have training set size (m) $>$ features size (n), i/e/ m $geq$ 10n for $\Sigma$ to be invertible.\\
                    \hline
                    
                \end{tabular}
                \caption{Anomaly detection: original model vs multivariate Guassian}
                \label{tab:anomaly-origin-multivariate}
            \end{table}
