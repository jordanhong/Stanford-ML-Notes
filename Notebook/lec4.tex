\section{Linear Regression with Multiple Variables}

    \subsection{Multiple features}
    Recall in the single variable case, we have a single input (x), two parameters($\theta_0, \theta_1$). The hypothesis can be expressed as: \[
        h_\theta(x)= \theta_0 + \theta_1x
    .\] 
  
    Now, consider a generalized case where there are multiple features: X\textsubscript{1}, X\textsubscript{2}, X\textsubscript{3}. The information can be organized in a table with example numerical values:
    \begin{table}[htbp]
            \begin{center}
                 \begin{tabular}{||c c c c||} 
                 \hline
                  Sample Number (i) & X\textsubscript{1} &  X\textsubscript{2} & y \\ [0.5eX] 
                 \hline\hline
                 1 & 6 & 87837 & 787 \\ 
                 \hline
                 2 & 7 & 78 & 5415 \\
                 \hline
                 3 & 545 & 778 & 7507 \\
                 \hline
                 4 & 545 & 18744 & 7560 \\
                 \hline
                 5 & 88 & 788 & 6344 \\ [1ex] 
                 \hline
                \end{tabular}
             \caption{Sample Table}
             \label{tab:data}
         \end{center}
     \end{table}


        From Table \ref{tab:data}, one can see that each row is a sample a feature on each column.

    \subsubsection{Notation}

        \begin{enumerate}
            \item \textbf{n}: number of features.
            \item \textbf{x\textsuperscript{(i)}}: (row vector) input features of the i\textsuperscript{th} training example. i= 1, 2,\dots, m. 
            \item \textbf{x\textsuperscript{(i)}\textsubscript{j}}: value of feature j in the i\textsuperscript{th} training example. j= 1, 2, \dots, n.  

        \end{enumerate}

    \subsubsection{Hypothesis}

        Previously, 
        \[ 
        h_\theta(x)= \theta_0 + \theta_1\cdot x 
        \]
    

        Now, we can extend the hypothesis to :

        \[
            h_\theta(x) = \theta_0\cdot1 + \theta_1\cdot x_1 + \theta_2\cdot x_2  
        \]

        For convenience of notation, let's define x\textsubscript{0}=1, i.e. x\textsuperscript{i}\textsubscript{0}=1 $\forall$ i.

        Therefore, we have: \textbf{x}= $\left[ \begin{array}{c}
                                                    x_0 \\
                                                    x_1 \\
                                                    x_2 \\
                                                    \vdots \\
                                                    x_n
                                                 \end{array}
                                          \right]$
                                          and \textbf{$\theta$} = $\left[ \begin{array}{c}
                                                    \theta_0 \\
                                                    \theta_1 \\
                                                    \theta_2 \\
                                                    \vdots \\
                                                    \theta_n \\
                                                 \end{array}
                                          \right]$. 
        Then, the hypothesis function can be written as:

            \begin{equation}
                \begin{split}
                 h_\theta (x) &= \left[ \begin{array}{ccccc}
                                                \theta_0 & \theta_1 & \theta_2 & \dots & \theta_n 
        \end{array} 
                                    \right] \cdot \left[ \begin{array}{c}
                                                                x_0 \\
                                                                x_1 \\
                                                                x_2 \\
                                                                \vdots \\
                                                                x_n
                                                                 \end{array} \right] \\
                                                                 & = \theta^T\cdot \textbf{x}
            \end{split}
        \end{equation}
        

        This is \emph{Multivariate linear regression}.










    \subsection{Gradient Descent for Multiple Variables}
        \subsubsection{Algorithm}
            \textbf{Summary for Multivariables} 
                \begin{enumerate}
                    \item \textbf{Hypothesis  }$h_\theta (x) = \theta^T \cdot \textbf{x}$
                    \item \textbf{Parameters } $\mathbf{\theta}$
                    \item \textbf{Cost Function } $ J(\mathbf{\theta}) = \frac{1}{2m} \sum_{i=1}^{m} (h_\theta(x^{(i)}) - y^{(i)} )^2 $
                    \item \textbf{Goal } $ \min_{\mathbf{\theta}} J( \mathbf{\theta}) $
                \end{enumerate}

        
                \textbf{Gradient Descent for Multiple Variables} \\
                
                        repeat until convergence \{  
                            \[ \theta_j := \theta_j - \alpha \frac{\partial }{\partial \theta_j} J(\mathbf{\theta})\mbox{\hspace{10pt} (for j=0, 1,$\ldots$ n)}  
                            \] 

                        \}  


                        
                       \[\theta_j := \theta_j - \alpha \frac{1}{m} \sum_{i=1}^{m} (h_\theta(\mathbf{x^{(i)}}) - y^{(i)} ) \cdot x_j^{(i)}\] \\ 
                       
                       Note: $x_0^{(i)} = 1$, by definition.
        
         \subsubsection{Vectorized Implementation}

         One can work out the linear algebra and arrive at the following simplification using vectorized operations.
            The cost function $J$ can be expressed as:
            \begin{equation}
                J(\theta) = \frac{1}{2m} (\mathbf{X\theta} - \mathbf{y})^T (\mathbf{X\theta} - \mathbf{y} )
                \label{eq:vectorized-cost}
            \end{equation}
            
            The MATLAB implementation is as follows:
            \begin{lstlisting}
                m = length(y); % calculate how many samples 
                J = 1/(2*m)*((X*theta-y).')*(X*theta-y);

            \end{lstlisting}


            Gradient descent can be vectorized in the form:
            \begin{equation}
                \theta = \theta - \frac{\alpha}{m} \cdot \mathbf{X}^T \cdot (\mathbf{X\theta} - \mathbf{y}) 
                \label{eq:vectorized-gradient-descent}
            \end{equation}

            The MATLAB implementation is as follows:
            \begin{lstlisting}
                m = length(y); % number of training examples
                for iter = 1:num_iters
                    theta = theta - alpha/m* X.'*(X*theta -y);
            \end{lstlisting}


    \subsection{Gradient Descent in Practise I: Feature Scaling} 
    
        \begin{itemize}
            \item Idea: ensure each featurre are on a similar scale
            \item Get every feature into approx. $-1\leq x_i \leq1 $($\sim$ order)
            \item \textbf{Mean Normalization}: Replace $x_i$ with $\frac{x_i - \mu_i}{s_i}$, where $\mu_i$ and $s_i$ are the sample mean and standard deviation, respectively.

        \end{itemize}

    \subsection{Gradient Descent in Practise II: Learning Rate}
        
        \begin{itemize}
            \item Ensure gradient descent is working: plot $J_\theta$ over each number of iteration (not over $\theta$ !)
            \item Example automatic convergence test: for sufficiently small $\alpha$ $J_\theta$ should decrease by less than $10^{-3}$ i one iteration.
            \item If $\alpha$ is too small, gradient descent can be slow to converge.
            \item If $\alpha$ is too large, gradient descent may not converge.
            \item To choose $\alpha$, try 0.001, 0.003, 0.01, 0.03, 0.1, 0.3, 1\ldots (by 3x)
        \end{itemize}


    \subsection{Features and Polynomial Regression}
        We can fit into different polynomials by choice, using multivariate regression. Recall
        \[
            h_\theta (x) = \theta_0 + \theta_1x_1 + \theta_2x_2 +\theta_3x_3 
        \]  

        Let x\textsubscript{1} be x\textsuperscript{1}, x\textsubscript{2} be x\textsuperscript{2}, x\textsubscript{3} be x\textsuperscript{3}.
        Note we should still apply feature scaling to x\textsubscript{1}, x\textsubscript{2}, and x\textsubscript{3} individually! 


    \subsection{Normal Equation}

        The normal equation provides a method to solve for $\theta$ analytically. For our data with m samples, n features, recall each sample can be written as: 
       \[ 
        \mathbf{x^{(i)}} =  \begin{bmatrix}
            x_0^{(i)}\\
            x_1^{(i)} \\
            x_2^{(i)} \\
            \vdots \\
            x_j^{(i)}\\
           \vdots \\
           x_n^{(i)}
            \end{bmatrix}
        \]
        We can construct a design matrix:
        \begin{equation}
            \mathbf{X} = \begin{bmatrix}
                \horzbar & (x^{(1)})^T & \horzbar \\
                \horzbar & (x^{(2)})^T & \horzbar \\
                \horzbar & (x^{(3)})^T & \horzbar \\
                         & \vdots      &          \\
                \horzbar & (x^{(m)})^T  & \horzbar \\
             \end{bmatrix}
             \label{eq:design-matrix}
        \end{equation}


         Then $\theta$ can be found by the normal equation: 
                 \begin{equation}
                 \mathbf{\theta} = \mathbf{(X^TX)}^{-1}\mathbf{Ty}
                     \label{eq:normal}
                 \end{equation}

        
        Normal equation is useful as no $\alpha$ is required to and we do not need to iterate. However, we do have to compute $(X^TX)^{-1}$, which can be computationally expensive when n is large. The complexity is O($n^3$) for inverse operations. Gradient Descent is useful when n is large (many features). 

     \subsection{Normal Equation and Non-invertibility}

     What if $(X^TX)^{-1}$ is non-invertible? 

     \begin{itemize}
         \item Redundant features (linearly dependent), i.e. having same information in two different units.
         \item Too many features (i.e. m $\leq$ n). Delete some features or use regularization
     \end{itemize}



