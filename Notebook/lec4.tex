\section{Linear Regression with Multiple Variables}

    \subsection{Multiple features}
    Recall in the single variable case, we have a single input (x), two parameters($\theta_0, \theta_1$). The hypothesis can be expressed as: \[
        h_\theta(x)= \theta_0 + \theta_1x
    .\] 
  
    Now, consider a generalized case where there are multiple features: X\textsubscript{1}, X\textsubscript{2}, X\textsubscript{3}. The information can be organized in a table with example numerical values:
    \begin{table}[htbp]
            \begin{center}
                 \begin{tabular}{||c c c c||} 
                 \hline
                  Sample Number (i) & X\textsubscript{1} &  X\textsubscript{2} & y \\ [0.5eX] 
                 \hline\hline
                 1 & 6 & 87837 & 787 \\ 
                 \hline
                 2 & 7 & 78 & 5415 \\
                 \hline
                 3 & 545 & 778 & 7507 \\
                 \hline
                 4 & 545 & 18744 & 7560 \\
                 \hline
                 5 & 88 & 788 & 6344 \\ [1ex] 
                 \hline
                \end{tabular}
             \caption{Sample Table}
             \label{tab:data}
         \end{center}
     \end{table}


        From Table \ref{tab:data}, one can see that each row is a sample a feature on each column.

    \subsubsection{Notation}

        \begin{enumerate}
            \item \textbf{n}: number of features.
            \item \textbf{x\textsuperscript{(i)}}: (row vector) input features of the i\textsuperscript{th} training example. i= 1, 2,\dots, m. 
            \item \textbf{x\textsuperscript{(i)}\textsubscript{j}}: value of feature j in the i\textsuperscript{th} training example. j= 1, 2, \dots, n.  

        \end{enumerate}

    \subsubsection{Hypothesis}

        Previously, 
        \[ 
        h_\theta(x)= \theta_0 + \theta_1\cdot x 
        \]
    

        Now, we can extend the hypothesis to :

        \[
            h_\theta(x) = \theta_0\cdot1 + \theta_1\cdot x_1 + \theta_2\cdot x_2  
        \]

        For convenience of notation, let's define x\textsubscript{0}=1, i.e. x\textsuperscript{i}\textsubscript{0}=1 $\forall$ i.

        Therefore, we have: \textbf{x}= $\left[ \begin{array}{c}
                                                    x_0 \\
                                                    x_1 \\
                                                    x_2 \\
                                                    \vdots \\
                                                    x_n
                                                 \end{array}
                                          \right]$
                                          and \textbf{$\theta$} = $\left[ \begin{array}{c}
                                                    \theta_0 \\
                                                    \theta_1 \\
                                                    \theta_2 \\
                                                    \vdots \\
                                                    \theta_n \\
                                                 \end{array}
                                          \right]$. 
        Then, the hypothesis function can be written as:

            \begin{equation}
                \begin{split}
                 h_\theta (x) &= \left[ \begin{array}{ccccc}
                                                \theta_0 & \theta_1 & \theta_2 & \dots & \theta_n 
        \end{array} 
                                    \right] \cdot \left[ \begin{array}{c}
                                                                x_0 \\
                                                                x_1 \\
                                                                x_2 \\
                                                                \vdots \\
                                                                x_n
                                                                 \end{array} \right] \\
                                                                 & = \theta^T\cdot \textbf{x}
            \end{split}
        \end{equation}
        

        This is \emph{Multivariate linear regression}.










    \subsection{Gradient Descent for Multiple Variables}

    Blah
